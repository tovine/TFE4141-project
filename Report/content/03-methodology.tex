\chapter{Methodology}
%This chapter should discuss the details of your implementation for the assignment. 
%Everything related to \emph{how} things were done should go here.
%Remember to avoid going into too much details, summarize appropriately and try to use figures/charts.
%Make sure you refer to the figures (such as Figure \ref{fig:universe}) and charts you add in the text.
%Avoid putting lots of source code here -- small code snippets are fine if you want to discuss something specific.
\section{Presentation of the solution}
We chose to implement the RSA circuit using one module for the MonPro operation, a datapath module with various registers for storing inputs and intermediate results and a top-level controller module with a state machine and control signals for the datapath.
In order to use Montgomery's algorithm without taking additional parameters, a simplified version of Blakley's algorithm was used to compute
\begin{equation}
    \bar{x}=1*R*mod(n), R=2^{128}
\end{equation}
\begin{equation}
    \bar{M}=M*R*mod(n), R=2^{128}
\end{equation}


\subsection{Blakley module}
In our case
The blakley module is implemented 

\subsection{Monpro module}


\subsection{Stitching everything together}

%\begin{figure}
%\centering
%\includegraphics[scale=1.7]{images/universe.jpg}
%\caption{A JPEG image of a galaxy. Use vector graphics instead if you can.}
%\label{fig:universe}
%\end{figure}
%%
%Add content in this section that describes how you tested and verified the correctness of your implementation, with respect to the requirements of the assignment.
\section{Verification plan}
- Write a verification plan.\\
- What metrics will you use to decide when you are done verifying?
(pass rate, code coverage, functional coverage).\\
- Demonstrate the use of assertions \\
- What bring up test strategy have you planned?\\ 
- Discuss/Analyze/Conclude\\
\\
Each submodule of the design was verified with its own testbench.
The RSAcore module was verified using the testbenches provided by Øystein Gjermundnes.
\\
Improvements in the verification plan would have been to use Bitvis library.